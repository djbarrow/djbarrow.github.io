\documentclass[a4paper,12pt,titlepage]{book}
\usepackage{url}
\usepackage{eurosym}
\author {(c)2005 Written by D.J. Barrow \\\\ barrow\_dj@yahoo.com \\\\
Inspired by others}  
\title{Mission to Mars}
\begin{document}
\maketitle
\chapter{Going to Mars, The future of space travel}
Space Travel \& Aviation is where we aren't keeping up
with science fiction \& most definitely not improved much
in the last 40 years.
Everybody thinks going to Mars \& back is much harder than going to the Moon.
There is one critical difference which makes it in some ways easier, Mars
has an atmosphere, jet engines will work there.
The downside is that the gravity on Mars is higher than the Moon's
the Moons gravity is about 16\% of Earth's, Martian
gravity is around 38\% that of Earts \& its escape velocity is
about 45\% that of Earth's at 11250 miles an hour.
The little Eagle Module was
enough to get off the Moon probably a 2 stage rocket
would be needed to get off Mars, this would be nearly
impossible to land on Mars \& this kind of payload
would require something larger than a Saturn V to
carry it to Mars \& it probably is impossible
to make using liquid propulsion.  
\paragraph
\noindent
Feynman first introduced the Rodger Ramjet idea of using an atomic powered
jet engine high in the Earth's atmosphere where it is very thin
to reach escape velocity \& slingshot to the planets.
For compactness in the design I
suggest to make a hybrid scramjet/rocket engine. When the
engine starts going initially it is a rocket \& it gradually
changes from a rocket to a scramjet as it picks up speed,
for the remainder of the document I am going to use
the term scramjet to mean a rocket/scramjet hybrid.
The great thing about an atomic scramjet is that it has almost
no theoretical upper speed limit except for the speed of light 
( I researched this fact quite a lot ) \& design
limitations because it has to carry no propellant with it
while in jet mode.
This idea was dumped because of it's environmental fallout
in the atmosphere not to mention in the crew compartment when it was
realized that atomic energy causes cancer.
Hopefully fear of environmental fallout 
will change when curing cancer will become easier in
the near future, see the next topic below.
The radiation in the crew compartment is the greatest
risk of all, this can be minimized by
moving the crew compartment as far away from the radioactive
core as possible \& putting a ``shadow'' shield as near
as possible to the core which creates a radiation ``shadow''
around the crew compartment which prevents neutrons, gamma rays
\& other radiation from entering the crew compartment from the
reactor core, This shadow shield will add a lot of weight
to the craft putting the propellant tank in the way also
will reduce the radiation risk to the crew further.
Because it probably would be a huge lump of lead,
I personally hate the idea of a shadow shield
\& if anyone has a more innovative idea I am all ears.
Ideally the scramjet should be using fusion as this
is environmentally friendly but this probably will be
impossible for another 50 years.
\paragraph
\noindent
The ``Greenpeace friendly'' \& technically easier but more expensive
approach would be to use
a Saturn V to carry the atomic scramjet to Mars
( this of course is only environmentally cleaner if the Saturn V 
doesn't explode so for safety it would be best to choose some
remote desert like the Sahara ),
seeing as the atomic scramjet would now only have to work in Mars's
atmosphere \& not have to work in Earth's it would be easier to design.
If Greenpeace allows a halfway house \& allows the atomic
scramjet operate high in the earths atmosphere it could
be piggybacked up by either
the American U2 spy plane or something bigger but similar to
scaled composites/Virgin Galactic's
Spaceship One launch vehicle provided it is powerful enough.
According to TM-1998-208834-REV1.pdf ( see below )
A Nuclear Thermal Rocket NTR ( \& therefore a scramjet ) 
would only use up around 10g of
enriched U-235 to leave the Earths atmosphere, seeing that
an atomic bomb like the one that exploded in Hiroshima
used around 11Kg of enriched U-235, the amount of
fallout generated by an atomic scramjet would only be
less that 0.1\% of the fallout generated by an atomic
bomb, surely this is acceptable even to Greenpeace.
Compared to other moronic proposed ideas like Project Orion
which was to ride the shock waves generated by exploding
atomic bombs behind the space ship this is a walk in the
park, this unfortunately isn't nice for getting to Mars
because it would take 8 over weeks minimum to get to Mars,
I personally would be hoping to do it in 2, this would use
16 times more fuel or 160g of Uranium in the Earths
atmosphere because
\begin{math} 
Energy=Mass x Velocity^{2} 
\end{math}
of course a chemical rocket if it could go that fast
would also have to burn 16 times as much fuel, but I believe
it can't go that fast anyway.
\paragraph
\noindent
To take off in the Martian atmosphere would be more difficult
that Earth's because the atmospheric pressure is ~6.1MB
( about 1/150th that of Earth ).
This means a lot of speed would need to be picked up before
the scramjet in jet mode \& aerodynamics work.
Maybe aerodynamic surfaces like wings should be avoided altogether
on some crafts if the weight impact isn't offset by the usefulness of them.
I would suspect that the scramjet would need
to reach somewhere between 5,000 \& 15,000 miles
an hour before it can switch fully from rocket mode
to scramjet mode. Because the Martian atmosphere is so thin
an adjustable funnel which can open very wide at low
speeds might be useful so that the scramjet can
operate in jet mode at lower speeds.
The great thing about an atomic 
scramjet is it needs to carry no propellant except for the 
bit of propellant to operate start off while in rocket mode.
It should be easy to get a unmanned small manless scramjet 
Mars for initial testing, an average sized rocket should do.
\paragraph
\noindent
A very sensitive long range radar \& an autopilot with
the ability to make minute course adjustments
under computer control may be needed to avoid
space debris which would be lethal at such high speeds,
\paragraph
\noindent
I would like the ship to be made as much as possible from
one solid piece of material \& be flexible enough change shape naturally
as aerodynamic stresses are applied to it. 
I personally don't like things that vibrate
vibrations typically cause things to break, for moving parts I like
things that go round \& round like turbines, they don't break much.
\paragraph
\noindent
For manless space probes hopefully no shadow shield will be
needed provided the electronics are fault tolerant to radiation.
This opens the possibility of a space probe going to Mars first,
then slingshotting inside the Martian atmosphere where the
environmentalists won't be giving out about fallout to a large
percentage of the speed of light \& heading off to Alpha
Centuari, this is possible with an atomic scramjet engine.
The difficult bit which I haven't figured out
is how to slow down the probe from the speed of light 
once it gets there if there are no planets with atmosphere's 
nearby of which
there are none around Alpha Centuari, The atomic core
could power a powerful radio transmitter which
would send back data to earth 4.31 years after arriving there.
If a probe cannot transmit receivable data over that
great distance maybe the scramjet plane could
get close enough into the edge of Alpha Centurai's atmosphere to
slingshot around the star \& return to the
solar system without burning up.
A test flight around the Sun could be done first
to check if the idea works.
\paragraph
\noindent
Videos of Alien abductions is the place to look for ideas for spaceships.
The more observant people will describe their experiences
accurately. An interesting concept from a video I watched
points the future for materials science, the materials
blended into each other without joints.
Anybody who watch ``Das boot'' knows that rivets pop
in World War II U-Boats at 250 meters \& glued Space
Shuttle heat shield tiles keep falling off not to mention O rings
which caused the 1984 Challenger space shuttle disaster,
Joints are from an engineering point of view bad ideas.
\paragraph
\noindent
Dynamic blends of material need to be made, blending smoothly
into each other, I suggest making alloys/dynamic blends of
titanium, carbon fiber, plastic, glass, fiberglass \& the ceramics
in space shuttle heat shields.
\paragraph
\noindent
To prevent loss of life the initial prototypes will be radio
controlled \& I hope John Mc Carmac would be interested in
writing the autopilot software to land the spaceship on Mars.
Owing to fifteen minute delays caused by radio waves getting to Mars \& back
the distance from Earth would be too far to land by remote control 
I also would hope that Scaled Composites could be made interested. 
\paragraph
\noindent
More propellant can be accumulated in the propellant tank
while flying by taking in some from the air intake of the
scramjet before leaving the atmosphere, if travelling
fast enough there may be enough gas in the vacuum
of space to keep the scramjet going in jet mode at
least some of the time.
\paragraph
\noindent
If some of the scramjet is made in Russia
maybe the Uranium 235 or Plutonium in some of the decommissioned
nuclear warheads can be used as fuel for the scramjets.
A parachute would make a good airbrake I would suggest
making the parachute in the scramjet
plane out of a strong metal capable of standing high
temperatures \& possibly with hinges
so it naturally packs away neatly,
but maybe a metal fabric is better.
An alternate method of braking in space would
be to turn the plane around 180 degrees \& fire
the rocket this could slow the ship down by
around 10,000 miles an hour but not more as the
rocket will run out of propellant,
this also cannot be done in the atmosphere
because the flames would be blown onto the
ship \& destroy it.
The ship I believe would need a protective magnetic 
field to prevent cosmic rays once we get outside
the Val Allen Belt.
\paragraph
\noindent
I got talking to a rocket scientist Bryan in NASA over email about
this project \& he pointed me at similar
proposals in NASA\\
Proposal Title Exploration of Jovian Atmosphere Using Nuclear Ramjet Flyer\\
Principal Investigator Maise, George\\
\url{http://www.niac.usra.edu/studies/study.jsp?id=510&cpnum=00-01&phase=II&last=Maise&first=George&middle=&title=Exploration%20of%20Jovian%20Atmosphere%20Using%20Nuclear%20Ramjet%20Flyer&organization=Plus%20Ultra%20Technologies,%20Inc.&begin_date=2001-03-01%2000:00:00.0&end_date=2003-01-31%2000:00:00.0}\\\\
The Mars nuclear airplane is discussed here:\\
Principal Investigator Powell, James\\
\url{http://www.niac.usra.edu/studies/study.jsp?id=424&cpnum=99-03&phase=I&last=Powell&first=James&middle=&title=Development%20of%20Self-Sustaining%20Mars%20Colonies%20Utilizing%20the%20North%20Polar%20Cap%20and%20the%20Martian%20Atmosphere&organization=Plus%20Ultra%20Technologies,%20Inc.&begin_date=2000-05-01%2000:00:00.0&end_date=2000-10-31%2000:00:00.0}\\\\
Vehicle and Mission Design Options for the Human Exploration of Mars/Phobos\\
Using "Bimodal" NTR and LANTR Propulsion\\
AUTHOR(S): Stanley K. Borowski, Leonard A. Dudzinski, and Melissa L. McGuire\\
\url{http://gltrs.grc.nasa.gov/reports/2002/TM-1998-208834-REV1.pdf}\\\\
High Power Nuclear Electric Propulsion (NEP) for Cargo and Propellant Transfer Missions in Cislunar Space\\
AUTHOR(S)Robert D. Falck and Stanley K. Borowski\\
\url{http://gltrs.grc.nasa.gov/reports/2003/TM-2003-212227.pdf}\\\\
He also found nothing obviously impossible in my proposal,
a mission to Mars \& even
Alpha Centauri is very feasible using todays technology.
This is Bryan's reply\\
Hello,\\\\
Please do not use my name.\\\\
Your suggested ideas really need a LOT more refinement...  The ramjet 
idea will fly in the atmosphere or Earth or Mars, but you still need to have 
propellants to get to Mars, and that is a LOT of propellant....  Please 
see the papers on NTR for a general idea...\\\\
Scooping all of that propellant out of the atmosphere and processing it 
into something usable for the interplanetary flight requires a pretty 
massive and complex system...\\\\
It is much simpler using a "traditional" NTR.\\\\
See you,\\\\
Bryan\\\\
What I figured out from this reply is that
we should be still able to get to Alpha Centuari
using a atomic scramjet in the Jovian Atmosphere.
\paragraph
\noindent
The reason that the space race to
the Moon happened was that John F. Kennedy suggested it \&
he had the clout to throw tons of money at the idea
national pride, fear of them damn commies \& space supremacy got involved to
generate the space race.
The trip to Mars should only cost only a tiny
fraction of the trip to the Moon if an atomic
scramjet could operate in both the Earths \& the Martian
atmosphere.\\
The reply I received from virgin galactic is below.\\\\
Subject: RE: Any interest in going to Mars?\\
Date: Mon, 21 Nov 2005 17:11:03 -0000\\
From:virgingalactic@virgin.com\\
Dear DJ\\\\
Thanks for your email.\\\\
We're still at the early stages of the project and very much focusing
all of our resources on getting it off the ground, so at this stage Mars
is a long way off.\\\\
Thanks so much for thinking of us and who knows, one day ...\\\\
Kind regards\\\\
The Team\\
Virgin Galactic\\\\
\paragraph
\noindent
By the time Virgin Galactic starts operating commercially
it will be 50 years since the first man Yuri Gagarin went into space
\& they won't even be going as high. They certainly
are not boldly going where no man has gone before
even commercially they are doing nothing new except
making a hop to the edge of space relatively cheap.
I personally suggest that you email \url{virgingalactic@virgin.com}
\& pester them till Richard Branson \& Burt Rutan \& the
others start listening to give serious consideration about going to
Mars using my suggestions \& the suggestions in the 
the proposals provided by NASA to make this happen.
Exploration of space needs to continue \& it might
be necessary for private enterprise to take over
from government agencies. If the Russians were
enterprising enough to take a space tourist
up they will gladly go with this project if
they get funding.
\paragraph
\noindent
Other people you can contact or pester:\\\\
George W. Bush: president@whitehouse.gov\\
Whitehouse switchboard 202-456-1414\\\\
Dave Fanning: dave@2fm.ie\\
He would be able to get Bono, Richard Branson \& Bob Geldof to listen.\\
Gerry Ryan:\\
Gerry Ryan Show\\
From Rep. Of Ireland: 1850 715 922\\
From N. Ireland or UK: 08457 585 285\\
(The Ryan Line is open Mon-Fri 9am-12)\\
Text: 087-772-0000\\
Email: grs@rte.ie\\
Gerry Ryan gerryryan@rte.ie
this email goes directly to Gerry Ryan's researcher.\\
Gerry Ryan would be able to get Richard Branson possibly
Bono, Bob Geldof \& Cmdr Collins the space shuttle astronaut
whom he interviewed around November 2005 to listen.
\paragraph
\noindent
I am going to set a goal for getting to Mars,
to 3 to 5 years time, see my open letter to Bob Geldof
Richard Branson \& others if you want to see how I hope
this will be achieved.

\chapter{Cures for cancer}
Some cancers can be cured by a virus, see
\url{http://www.onoclyticsbiotech.com} for more info.
Surely this is the way to go as opposed to chemotherapy.
How about some preventative medicine imagine if children
were immunized against cancer before they got it or
even having the virus put into the foodchain atomic
power would become quite safe.
As Kristian Walsh pointed out radiation causes more
damage than just cancer the question I ask is this
damage acceptable. It would also be worth
researchers while to figure out why the animals
living near Chernobyl are surviving \& make use of
the information.

\chapter{Spacesuits}
These wouldn't be necessary for the mission to Mars but
if we manage to mitigate the cosmic rays
causing cancer problem by the anti cancer virus
in the future how about Spacesuits rather than 
weighing around 650lbs instead made so tight
they prevent the body exploding owing to pressure 
outside the suit, these would not need atmospheric pressure 
built up inside them \& be more like high altitude skydiving suits.
The only thing I am not sure about is whether the spacesuit
could be made to have insulation properties \& have heating
equipment installed to survive the temperature extremes of space.
\chapter{Open letter to NASA, Gerry Ryan, Bob Geldof, Bill Gates, Bono,\\
Steve Jobs, Larry Ellison, Michael O' Leary, Eddie Jordan, Richard Branson,\\
Burt Rutan John Mac Carmac, Paul Allen \& Michael Jackson}
\paragraph
\noindent
Bill Gates I know you love Richard Feynman, so do I.
I know you love Porches \& the film The Clockwork Orange
my favourite film is Dangerous Liaisons, it's better,
any interest in a Dangerous Liaison with me?
Bill you are worth more than the GDP of many countries
Why not give half your money to Africa \&
your kids won't be spoilt if they
learn the value of money, stop overprotecting them.
You can put a tiny bit of the rest of it in the Mission to
Mars Project, you will still be a billionaire 
many times over anyway \&
still worth more than the GDP of many countries.
What the hell do you want it all for anyway?, you can't
take it with you when you die.
\paragraph
\noindent
Bob Geldof, Bono, Michael Jackson how would you
like to hold a leg of Live Aid 2 from Mars.
Michael Jackson you are a nice guy.
I heard you were interested in meeting
Martians, we can make the guys who go up in the
scramjet citizens of Mars, would that do you?
\paragraph
\noindent
Larry Ellison I know you love MIGs what do
you think of my Bat out of hell idea would
you like to go up?
The Scramjet, if aerodynamics prove useful for it in the Martian
atmosphere may perform like a fighter but would be much faster
to get you home from Mars in time for tea, the atomic engine should last
years while flying inside the Martian atmosphere
\& we can put an autopilot on it so you can sleep \& you only
have to come home when you get bored.
Steve Jobs have any interest in dropping the rollerblades \& getting
involved?, what about you Richard?
Michael O' Leary CEO of Ryanair have you any interest in keeping
control of finances to make sure the job gets done cheaply?,
Eddie Jordan have you any interest in organizing the advertising?
The only thing I would like painted on the thing myself is a Swastika
with a Star of David next to it symbolizing peace between
Germans \& Jews \& possibly a bit of Graffiti.
Gerry Ryan can you get Bob Geldof, Bono \& Cmdr. Collins
the NASA Astronaut who recently went up on the Space Shuttle
whom you interviewed recently going so he can get NASA interested,
\& make Live Aid2 a reality.
\paragraph
\noindent
I love Russian Tu-144 jets,
they were made of primitive technology \& were
300 miles an hour faster than Concorde.
If the craft is being built by private enterprise
it would be cheapest to build it in Russia.
While the Americans have their spaceships
in the Smithsonian Museum the Russians have
theirs rotting in the back yards of their launch pads, it's sad.
They Russian's space agency would really appreciate the money
provided by private enterprise they really
deserve a shot to help make this happen.
In any case I would like the other space agencies
like NASA to get involved too.
\paragraph
\noindent
Bill I know you are interested in beating Linux at
it's own game, here is what you do... support Cygwin.
You can already run nearly all the Linux apps on Windows.
Steve Jobs has got Linux engineers flying to MacOS X
in droves because there are about 50 gui's on Linux
hardly any of them any good except KDE. Windows is safe
your users don't want the headache of learning Linux.
KDE boys your software is far too expensive on Windows.
why don't you make your software MYPL
'pay us what you think the software is worth',
In an ideal world everything should be under MYPL even food.
If KDE is any good people will pay you for it rather
than see it die. I personally hope that all the open
source GUI developers will stop reinventing the wheel
\& go with KDE \& give us a consistent interface for Linux.
Bill I think all software should be sold under MYPL
All my software is officially under GPL but
unofficially I hope to be paid for it. I am a communist
rather than a capitalist unlike yourself who likes
haggling, I couldn't be arsed running around looking
for money off people who really don't want to give it to me
possibly because they don't think my software is much good anyway
\& chasing ``software pirates'' it wastes too much energy.
\paragraph
\noindent
Sincerely,
D.J. Barrow
\end{document}
